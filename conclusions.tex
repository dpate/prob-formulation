\section*{Conclusions}
In this paper, we have presented an approach for describing  MDAO problems with 
with a graph based syntax.  Our graph description shares 
similarities to other approaches such as REMS, FDT, DSM, and XDSM, but 
it provides new constructs tailored to applications in early phases of the design 
process where a problem formulation has not yet been fully formed.   
In particular, we introduce the concepts of the Maximal Connectivity Graph (MCG) 
and the Fundamental Problem Graph (FPG).  

The MCG addresses the question, ``Given a set of analysis tools and design objectives, what are all 
of the potential problem formulations that could exist to solve a design problem?''  
The MCG provides a structured formalism to reduce the, potentially very large, 
set of valid combinations of analysis tools to a single subset and to identify
conflicts that could prevent a valid problem formulation.
Conflicts can originate from two sources: holes and collisisons. A hole 
corresponds to a variable that is not calculated by any analysis block in the 
MCG but is needed by other blocks. To create a valid problem formulation, all 
holes must be ``filled'' either by selecting the corresponding variable as a 
design variable, or by introducing additional analysis tools to compute the variable.  
A collision happens when two or more analysis tools compute values for the same variable.  
To create a valid problem formulation, all collisions must be resolved either by 
removing redundant analysis tools or by explicitly allowing the redundencies to remain 
in the form of a multifidelity problem. For the latter case, a special solution 
method will need to be employed to manage the multifidelity situation and decide 
when to execute the different analysis tools. Resolving collisions and filling 
holes in an MCG both require choices by the designer to shape the overall
MDAO problem to meet the needs of the problem he/she is trying to solve. 

The FPG represents the problem as an engineer would describe it: ``Given a specific set of tools, 
with a set of design vairables, find design that best meets a given objective
within the given constraints.'' Although an FPG uniquely defines a single MDAO problem, 
there is not necessarily a unique FPG for any given MCG. In Section 
\ref{s:building graphs} we provide an algorithm 
that reduces an MCG, via the resolution of conflicts in the graph, down to an FPG. 
How one goes about resolving the conflicts will determine which particular FPG is reached. 
One benefit of using graph theory is the standard techniques, such as cycle 
detection, which can be used to inform the user's decisions and make the process of conflict 
resolution much easier. 

In Section \ref{s:example problem}, we provide an example of 
formulating different FPGs from an MCG for a commercial aircraft design problem. This problem includes 8 
analysis tools and approximately 20 variables. Despite this relatively modest size, 
the MCG is rich enough to provide for a number of different valid FPGs. This demonstrates
the inherant value of the MCG to the design process by providing a mechanism for selecting 
an FPG based on important factors such as available design cycle time, consideration of 
coupling effects, or desired fidelity. Even after downselecting to a single FPG, the 
MCG itself still remains a useful tool. No real design problem is static, and will always
evolve throughout the design process. New tools may be introduced to account for previously 
unexpected effects and new constraints or objectives can be introduced. When this occurs, 
the MCG provides the opportunity to account for these changes by starting over with the 
algorithm from Section \ref{s:building graphs} to downselect to the most effective new FPG. 
While it is possible that this new FPG will be very similar to the previous one, 
it is also very possible that it will be different in significant ways. 
By repeating downselection algorithm from the MCG you allow for any new interdisciplinary couplings and  
information conflicts to play a role in selecting the most appropriate FPG. 

Once a desired FPG is identified, the driver node and driven edge described in the paper can be 
leveraged to formulate a Problem Solution Graph (PSG). This graph includes all of the 
same information from the FPG while additionally describing the particular MDAO 
solution architecture, e.g. MDF, IDF, BLISS, ATC, that will be used to solve the problem.  
Many possible PSGs could be formulated to solve a problem corresponding to a 
specified FPG. The application of the syntax to PSGs has not been illustrated in the paper, 
but the basic concept of applying multiple solution strategies to a given problem 
formulation has been well established by a number of MDAO architecture 
benchmarking efforts. Furthermore the validity of applying 
graph syntax to the description of specific problem solutions has been demonstrated 
by XDSM. 

For simple problems with few analysis tools and variables, the formulation of a 
valid problem description is straightforward.  However, MDAO problems continue to 
grow in scale and complexity. As the numbers of analysis tools and variables have 
increased, it has become increasingly challenging and time consuming for engineering 
teams to determine how the multiple analysis tools can be interconnected to produce 
valid problem formulations. Large complex problems also present signficant challenges
as the design process evolves requiring the integration high-fidelity 
analysis tools into an existing model. The graph syntax presented in this paper 
is intended to offer value in this context of increasing problem complexity by 
providing a formal approach to identify and resolve data conflicts in a set of interconnected
analysis tools and to help arrange those tools into an effective and efficient system model
where MDAO tools and techniques can be quickly and easily applied.  


