% IDF architecture diagram produced by the TikZ package
\documentclass{article}
\usepackage{geometry}
\usepackage{amsfonts}
\usepackage{amsmath}
\usepackage{amssymb}

\usepackage{tikz}

\input{diagram_border}

\begin{document}

% Style definition outside any picture
\input{diagram_styles}

\begin{tikzpicture}

	% Use a matrix to line up all the nodes 
    \matrix[MatrixSetup]{
		% First row
		& \node [DataIO](input_x){$x_1^{(0)},z_1^{(0)},z_2^{(0)}$};
		&
		& \node [DataInter](input_y){$\hat{y_2}^{(0)}$};
		&
		& 
		&
		\\
		% second row
		\node [DataIO] (x_output) {$x_1^*,z_1^*,z_2^*$};
		& \node [Optimization] (opt) {\MultilineComponent{2.3cm}{0,7$\rightarrow$1:}{Optimization}};
		& 
		& \node [DataInter](opt-x1){$1:x_1,z_1,z_2$};
		& \node [DataInter](opt-x2){$1:z_1,z_2$};
		& 
		& \node [DataInter](opt-x3){$1:x_1,z_2$};
		\\
		% third row
		& 
		& \node [MDA] (mda) {\MultilineComponent{1.5cm}{1, 5$\rightarrow$2:}{MDA}};
		& \node [DataInter](mda-y2){$2:\hat{y_2}$};
		& 
		& \node [DataInter](mda-y2_2){$2:\hat{y_2}$};
		& 
		\\
		% fourth row
		\node [DataIO] (y_output) {$y_1^*$};
		& 
		& 
		& \node [Analysis] (D_1) {\MultilineComponent{1.6cm}{2:}{$D_1$}};
		& \node [DataInter](mda-y1){$3:y_1$};
		& 
		& \node [DataInter](mdadone-y1){$6:y_1$};
		\\
		% fifth row
		\node [DataIO] (y2_output) {$y_2^{*}$};
		& 
		& 
		& 
		& \node [Analysis] (D_2) {\MultilineComponent{1.6cm}{3:}{$D_2$}};
		& \node [DataInter](mda-y2){$4:y_2$};
		& \node [DataInter](mdadone-y2){$6:y_2$};
		\\
		% sixth row
		& 
		& \node [DataInter](mda-g){$5:h$};
		& 
		& 
		& \node [Function] (constraints) {\MultilineComponent{1.6cm}{4:}{H}}; 
		& 
		\\
		% seventh row
		& \node [DataInter](opt-f){$7:f,g_1,g_2$};
		& 
		& 
		& 
		& 
		& \node [Function_i] (objective) {\MultilineComponent{1.6cm}{6:}{F,$G_1$,$G_2$}}; 
		\\
	};
	
	% Now define edges to tie the nodes together, using chains	
	{ [start chain=process]
		\begin{pgfonlayer}{process}
		\chainin (input_x);
		\chainin (opt)		[join=by ProcessTip];
		\chainin (mda)		[join=by ProcessHV];
		\chainin (D_1)		[join=by ProcessHV];
		\chainin (D_2)		[join=by ProcessHV];
		\chainin (constraints)		[join=by ProcessHV];
		\chainin (mda)		[join=by ProcessHV];
		\chainin (objective)	[join=by ProcessHV];
		\chainin (opt) [join=by ProcessHV];
		\chainin (x_output)	[join=by ProcessTip];
		\end{pgfonlayer}
	}
	
	\begin{pgfonlayer}{data}
	% Vertical edges
	\path (input_x) edge [DataLine] (opt-f)
	(input_y) edge [DataLine] (D_1)
	(input_x) edge [DataLine] (opt)
	(opt) edge [DataLine] (opt-x1)
	(opt) edge [DataLine] (opt-x2)
	(opt) edge [DataLine] (opt-x3)
	(D_1) edge [DataLine] (mda-y1)
	(D_1) edge [DataLine] (mdadone-y1)
	(D_2) edge [DataLine] (mda-y2)
	(D_2) edge [DataLine] (mdadone-y2)
	(D_2) edge [DataLine] (opt-x2)
	(constraints) edge [DataLine] (mda-g)
	(constraints) edge [DataLine] (mda-y2)
	(constraints) edge [DataLine] (mda-y2_2)
	(mda-g) edge [DataLine] (mda)
	(objective) edge [DataLine] (opt-f)
	(objective) edge [DataLine] (opt-x3)

	(y_output) edge [DataLine] (D_1)
	(y2_output) edge [DataLine] (D_2)
;
	\end{pgfonlayer}
	
\end{tikzpicture}

\end{document}