% IDF architecture diagram produced by the TikZ package
\documentclass{article}
\usepackage{geometry}
\usepackage{amsfonts}
\usepackage{amsmath}
\usepackage{amssymb}

\usepackage{tikz}

\input{diagram_border}

\begin{document}

% Style definition outside any picture
\input{diagram_styles}

\begin{tikzpicture}

	% Use a matrix to line up all the nodes
	\matrix[MatrixSetup]{
		% First row
		& 
		\node [DataIO] (Input) {$x,y^{t}$}; & 
		& 
		\\
		% Second row
		\node [DataIO] (Output) {$x^*$}; & 
		\node [Optimization] (Opt) {\MultilineComponent{2.3cm}{0,3$\rightarrow$1:}{Optimization}}; &
		\node [DataInter_i] (Opt-DA1) {$1:x_0,x_i,y_{j \neq i}^t$}; & 
		\node [DataInter] (Opt-Func) {$2:x,y^t$}; \\
		\node [DataIO_i] (Output_y1) {$y_i^*$}; & 
		&% Third row
		\node [Analysis] (DA1) {\MultilineComponent{1.6cm}{1:}{A}}; &
		\node [Analysis] (DA1) {\MultilineComponent{1.6cm}{1:}{A}}; &

		\node [DataInter_i] (DA1-Func) {$2:y_i$}; \\
		% Fourth row
		& 
		\node [DataInter] (Func-Opt) {$3:f,g,g^c$}; &
		& 
		\node [Function] (Obj) {\MultilineComponent{1.8cm}{2:}{f}}; \\
		\node [Function] (Const) {\MultilineComponent{1.8cm}{2:}{G}}; \\
	};
	
	% Now define edges to tie the nodes together, using chains	
	{ [start chain=process]
		\begin{pgfonlayer}{process}
		\chainin (Input);
		\chainin (Opt)		[join=by ProcessTip];
		\chainin (DA1)		[join=by ProcessHV];
		\chainin (Func)		[join=by ProcessHV];
		\chainin (Opt)		[join=by ProcessHV];
		\chainin (Output)	[join=by ProcessTip];
		\end{pgfonlayer}
	}
	
	\begin{pgfonlayer}{data}
	% Vertical edges
	\path (Input) edge [DataLine] (Func-Opt)
	(Opt-DA1) edge [DataLine] (DA1)
	(Opt-Func) edge [DataLine] (Func)
	% Horizontal edges
	(Output) edge [DataLine] (Opt-Func)
	(Output_y1) edge [DataLine] (DA1-Func)
	(Func-Opt) edge [DataLine] (Func);
	\end{pgfonlayer}
	
\end{tikzpicture}

\end{document}