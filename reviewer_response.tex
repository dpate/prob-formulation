\documentclass{aiaa-tc}

\title{reviewer responses for: \\ A Graph Theoretic Approach to Problem Formulation for 
Multidisciplinary Design Analysis and Optimization}

\usepackage{color}

\newenvironment{rev}{\vspace{2em}\itshape \textcolor{red}}{}


\begin{document}
\maketitle

\section{Reviewer1}

\rev{As the authors point out the design of a major product, such as a civil airliner, 
is complex and the level of complexity increases as design requirements change i.e. the 
need to reduce carbon emissions, the need to reduce maintenance downtime etc. and the 
number of design teams involved in creating a new product increases in number and 
distribution. Their paper is, I think, attempting to address the need to support 
design teams in confronting this situation but have not clearly defined where their work 
fits into the design process \\\\
Broadly speaking there are three major tasks or steps that have to be worked 
through when a globally distributed team is setting up a design system that allows it to 
effectively deploy an MDO methodology to support the design of a product such as a new 
civil airliner: 
\\\\1. The team must decide which design tools are required at each stage of the design 
process as the design proceeds down the time-line this includes the data flow required 
between the various design centers and the design history preservation requirements,
\\\\2. This combination of requirements must be converted into a set of interacting tools 
together with data flow and control programs that constitute a template for a 
Computational Design Engine (CDE),
\\\\3. Finally this template has to be instantiated into a usable CDE software system 
which might take advantage of commercially available software such as that found in Simula etc.
\\\\The paper is clearly focused on step 2 above and it advances a sound argument for the use 
of graph theory to establish an appropriate set of outcomes from such a step and, whilst 
it implies the existence of the other two steps, it does not state this early enough in 
the paper and with sufficient detail to allow a practicing engineer to understand the 
role being played by the authors contribution. I would recommend that the authors clearly 
define the role that their graph theoretic approach plays in this triple set of steps 
leading to the creation of effective CDE.}

The reviewer has correctly identified the major motivation for this work. The goal is indeed to 
assist engineers and teams of engineers in the process of building a model out of a set of interconnected 
tools. 

We believe that the distrinction between phases 1 and 2 from the comments are covered adequately 
in the bulk of the introduction and especially in the notional problem described there in. 

However, to clarify the disctinction between phases 2 and 3 we hav added the following near 
line 52, page 3 of the original submission: 

\emph{%
Our goal is to develop formalism for expressing analysis interconnectivity and for determining feasible
    analysis tool sets to assist an engineering team conducting this task. Because the problem deals with
    interconnectivity, we base our approach on the representations and techniques of graph theory.
    The approach begins by constructing the \emph{maximal connectivity graph (MCG)} describing all possible
    interconnections between the analysis tools proposed by the engineers. Graph operations are then
    conducted to reduce the MCG to a \emph{fundamental problem graph (FPG)} that describes the set of analysis
    tools needed to solve the specified system-level design problem. The concept of the FPG and the identification of feasible FPGs from an MCG are the main contributions of the paper. \\\\
    This information in an FPG represents the engineering design problem that needs to be solved, but it is not 
    itself sufficient to actually run a design optimization. It does not provide infomration 
    about how to solve the problem. Even after the problem is defined an integration platform or framework needs 
    to be selected and appropriate optimization methods identified. This last step essentially 
    involves taking the FPG and turning it into a usable model. We breifly consider methods to further 
    manipulate the FPG into a \emph{problem solution graph (PSG)} which could be usefull in this 
    task, but that is not the focus of this paper. The work is primarily concerned with applying graph 
    theory to the creation of an FPG and the benefits to the design process by doing so.}

\end{document}