\section{Introduction}
    \label{s:intro}
    The number of analysis tools required in multidisciplinary design optimization (MDO) studies is growing
    in parallel with the increasing scope of typical problems. An example can
    be observed in the historical evolution of the disciplines involved in MDO problems in aircraft design.
    Multidisciplinary optimization emerged as a separate field from structural optimization through the
    need to introduce formal techniques for managing the coupling of aerodynamic loads and structural
    deformations, through the linking of aerodynamic vortex lattice or panel methods with structural finite
    element models \cite{Cramer1994}. Subsequently, flight performance and life cycle economics tools were integrated
    into MDO analysis workflows for conceptual and preliminary design studies \cite{Sobieski1998}. Currently, MDO
    problems for aircraft design also often include tools for aircraft noise and emissions. In sum, it is now
    commonplace for 5--10 analysis tools to be employed for typical aircraft design optimization studies.

    The number of analysis tools is expected to grow in the future as the scope of MDO problems continues
    to evolve, and as computing is increasingly commoditized. This expansion of scope will be driven, in
    part, by consideration of additional disciplines. Current trends on the horizon for aircraft MDO studies
    include incorporation of manufacturing analyses \cite{deWeck2007}, subsystem performance \cite{Dean2009,Gavel2006}, and models of emissions, noise, or economics \cite{Antoine2004,Rallabhandi2007,Kirby2008}.

    Multidisciplinary Design Analysis and Optimization (MDAO)
    frameworks such as iSight\textsuperscript{\textregistered},\\ ModelCenter\textsuperscript{\textregistered}, ModeFrontier\textsuperscript{\textregistered}, and OpenMDAO\cite{Gray2012} have enabled a new level of analysis tool integration
    and have paved the way for models with more analyses and increasing numbers of interdisciplinary couplings. This
    new capability for tool integration has created a new challenge: Even models with tens of analyses could include hundreds or thousands of variables that are interdependent and must be linked in the framework.

    A common situation is that different analyses provide differing predictions for the
    same physical quantity, often in different data formats, and these conflicts need to be resolved.
    These occurrences are particularly acute in situations in which analysis tools have differing fidelities. For
    example, an abstracted aerodynamic analysis such as an empirical drag buildup model may return only
    integrated drag, whereas a CFD tool may return pressure and shear stress distributions across the entire
    surface grid. If an analysis downstream of the aerodynamics tool needs only integrated drag as an input,
    then the designer has a free choice of which of the two possible aerodynamics analysis tools to select to
    provide the drag estimate (presuming that drag can be computed from surface distributions by a simple
    integration algorithm). On the other hand, if a downstream analysis needs a pressure distribution in
    order to compute pitching moment, for example, then any feasible MDO problem formulation must
    include CFD or similar analysis in the data flow, regardless of whether the empirical drag buildup model
    is also included.

    With the added complexity from larger models, it is plausible that the task of combining all the analyses into a
    consistent system model capable of solving a relevant engineering design
    problem could approach the cost and time requirements of creating any of the discipline
    analyses themselves. For these large scale MDO problems, the couplings between the
    analyses begin to dominate the effort required in setting up the model. It is this problem of
    determining sets of analysis tools and their inter-connectivities to form realistic
    multidisciplinary problems that is the subject of this paper. We are motivated by the following notional
    but realistic problem of organizing an MDO study for a complex system:

    \begin{quote}
    A new system is being designed for which there is little or no historical precedent. The system
    is complex, as measured by the number of coupled disciplines and/or components involved in the
    analysis. A general optimization problem statement has been formulated based on system-level objectives and constraints; however, it is unclear which engineering analysis tools should
    be interconnected in order to solve the optimization problem. A team of disciplinary and/or component design engineers has been formed in which each engineer has expertise in a
    particular analysis tool or component model. The engineers meet to discuss the approach to
    interconnecting their tools to achieve the required system-level MDO model.
    \end{quote}

    Our goal is to develop formalism for expressing analysis interconnectivity and for determining feasible
    analysis tool sets to assist an engineering team conducting this task. Because the problem deals with
    interconnectivity, we base our approach on the representations and techniques of graph theory.
    The approach begins by constructing the \emph{maximal connectivity graph (MCG)} describing all possible
    interconnections between the analysis tools proposed by the engineers. Graph operations are then
    conducted to reduce the MCG to a \emph{fundamental problem graph (FPG)} that describes a set of analysis
    tools that collectively specify a system-level design problem. The concept of the FPG and the identification of feasible FPGs from an MCG are the main contributions of the paper.

    The information in an FPG represents the engineering design problem that needs to be solved, but the FPG does not
    itself provide sufficient information to run a design optimization. In particular, it does not provide information
    about \textit{how} to solve the problem. Even after the problem is defined, an integration platform or framework needs
    to be selected and appropriate optimization methods identified. This last step essentially
    involves converting the FPG into a usable model. We briefly consider methods to further
    manipulate the FPG into a \emph{problem solution graph (PSG)} which could be useful in this
    task, but the PSG is not the primary focus of this paper. Rather, the work is primarily concerned with applying graph
    theory to the creation of an FPG and the benefits to the design process obtained by doing so.

    The paper is organized as follows. First, we describe the differences between a fundamental problem
    formulation, which is based only on the system-level optimization problem statement that the
    engineers desire to solve and on the available analysis tools, and a specific problem formulation, which
    additionally presumes a specific MDO solution approach to the problem. Next, we survey the literature related to
    graph theoretic and formal language approaches to multidisciplinary design problem formulation.
    We then discuss our graph syntax and representation of MDO problems and describe the procedures for
    determining the MCG and FPG. Finally, we present an example problem based on an MDO analysis of a
    commercial aircraft.

\section{Background}
\subsection{Specific vs Fundamental Problem Formulation }
	\label{s:specific vs fundamental}
    The Fundamental Problem Formulation (FPF) is our terminology for a statement of the overall system-level optimization problem that contains only  information about analysis tools, design variables, objectives, and constraints without reference to a solution approach.
In particular, the FPF does not depend on the choice of solution-specific elements such as the MDAO framework, optimization algorithm, iterative solver, or execution sequence.
This description represents  the problem from the point of view of an engineer specifying the design problem that he/she desires to solve without specifying how the problem should be solved.
The lack of solution information in the FPF contrasts with optimization problem statements that are written with reference to a specific solution strategy or with implication of a specific execution sequence.
For example, consider the Sellar test problem \cite{AIAA:sellar} with an FPF given as follows:
    \begin{align}
        \txt{given} & \ \ y_1 = D_1(x_1,y_2,z_1,z_2) \notag
        \\      & \ \ y_2 = D_2(y_1,z_1,z_2) \notag
        \\\txt{min.} &\ \ F(x_1,y_1,y_2,z_2) \notag
        \\\txt{w.r.t.} & \ \ x_1,y_1,y_2,z_1,z_2 \notag
        \\\txt{s.t.} & \ \ G_1(y_1) \geq 0 \notag
        \\     & \ \ G_2(y_2) \geq 0
        \label{eqn:simple_fpf}
    \end{align}
    In Eq.~\ref{eqn:simple_fpf}, $y_1$ is an output of $D_1$ as well as an input
    to $D_2$. Similarly, $y_2$ is an output of $D_2$ as well as an input
    to $D_1$. This structure implies that $D_1$ and $D_2$ are coupled together by means
    of their mutual dependence. However, no information is given as to whether
    $D_1$ or $D_2$ should be executed before the other. Given Eq.~\ref{eqn:simple_fpf},
    it would be equally valid to execute $D_1$ first, $D_2$ first, or both simultaneously.
An MDAO root finding method or compatibility constraint formulation could be implemented to achieve consistency between $D_1$ and $D_2$.

    The Sellar problem is simple, with only a single coupling interaction
    between the two disciplines and a very limited set of variables.
 However, for larger and more complex problems,
    it is much more difficult to identify the fundamental problem formulation. Interdisciplinary
    couplings are not always apparent and the set of analysis tools and variables are
    much larger.

An example of a more specific problem statement for the Sellar problem is presented in Eq. 2.
In this formulation, the implication is that  $D_1$ must be run
    first. A new variable, $\hat{y_2}$, is introduced to break the direct dependence of
    $D_1$ on $D_2$, and a new coupling constraint, $H$, is added to enforce
    consistency. Equation \ref{eqn:simple} is
    an equally valid expression of the Sellar problem that could be
    generated  based on a preferred solution approach.
    \begin{align}
        \txt{given} & \ \ y_1 = D_1(x_1,\hat{y_2},z_1,z_2) \notag
        \\      & \ \ y_2 = D_2(y_1,z_1,z_2) \notag
        \\\txt{min.} &\ \ F(x_1,y_1,y_2,z_2) \notag
        \\\txt{w.r.t.} & \ \ x_1,y_1,\hat{y_2},z_1,z_2 \notag
        \\\txt{s.t.} & \ \ H(y_2,\hat{y_2}) = 0 \notag
        \\     & \ \ G_1(y_1) \geq 0 \notag
        \\     & \ \ G_2(y_2) \geq 0
        \label{eqn:simple}
    \end{align}
\subsection{Background in Graph-Based Descriptions of MDAO Problems}
	\label{s:existing syntax}
    As indicated in the examples in Sec.~\ref{s:specific vs fundamental}, the mathematical
    language for specifying optimization problem formulations is very general and can be used both for
    fundamental and specific problem formulations. Tedford and Martins used this syntax to specify the
    FPF for a set of test problems and also to describe specific formulations for solving them with a
    number of optimization architectures \cite{Tedford2009}. By solving different specific problem statements corresponding to the same FPF, they were able to  benchmark the performance of different optimization architectures against a fixed set of
    problems. Gray et al. also benchmarked the performance of a set of MDAO architectures  with a similar approach \cite{Gray2013}.
    This work examined a larger set of architectures and proposed the use of OpenMDAO as a platform to build a larger community-developed set of test problems and architectures. Both of these benchmarking efforts
    demonstrate how multiple specific problem formulations can relate to a common FPF and indicate the
    value of a common problem description. The challenge with this
    traditional syntax is that it is not easily manipulated or analyzed with automatic procedures to explore alternate problem formulations.

    A number of graph-based methods have been used successfully to translate the
    mathematical syntax into a more useful computational form.
    Steward's Design Structure Matrix (DSM) is a square adjacency matrix which captures the relationship between analysis tools \cite{Steward1981}.
    Off-diagonal elements of the matrix indicate coupling. Since a DSM describes a square adjacency matrix,
    it can be represented as an equivalent directed graph in which nodes represent analysis tools and
    edges represent information dependence between those tools. The ordering of elements in a DSM can be used to indicate
    execution order.  For more complex problems, choosing the proper order to run analysis tools is a challenging task.
    Rogers et al.~developed DeMAID to manipulate a DSM to find an ordering for analysis tools that
    reduces the cost of solving highly coupled systems \cite{rogers1996,rogers1996demaid}. This re-ordering is done through
    operations on the DSM matrix and yields multiple specific problem
    formulations which all solve the same FPF.

    A DSM itself is insufficient to describe complete optimization problem formulations because it
    captures only information about data dependency between analyses; objective and
    constraint information is missing from the DSM description of the problem.
    An alternative matrix-based syntax, called a Functional Dependency Table (FDT), was proposed by Wagner and Papalambros \cite{Wagner1993}.
    The FDT represents the relationship between functions, including objectives and constraints, and specific variables that affect
    them. Similar to DSM, FDT also describes an adjacency matrix of a graph. Unlike the DSM graph,
    however, the FDT graph is undirected and nodes can represent analysis tools, objectives,
    or constraints. Edges between nodes represent a dependence on the same
    variable. Michelena and Papalambros made use of the FDT to solve a graph partitioning problem that yielded
    more efficient optimization problem decompositions \cite{Michelena1997}. Allison
    extended this work to incorporate adjacency matrix information with the FDT to
    account for system coupling in an automated partitioning scheme \cite{Allison2008}.

    Although FDT succeeds at capturing the
    information about objectives and constraints, its lack of directed edges
    implies that it cannot describe the coupled data dependency that a DSM captures.
    Table \ref{t:FDT_simple}  shows the FDT for the Sellar problem described by Eq.~\ref{eqn:simple_fpf}.
    The FDT indicates that $D_1$ is dependent on $y_2$ and
    that $D_2$ is dependent on $y_1$, but the coupled dependence cannot be inferred from
    that information alone. This missing information implies
    that, although FDT is very useful for partitioning problems, it is not
    sufficient to completely describe the data flows in a problem formulation.

\begin{table}[htb!]
  \centering
        \caption{Sellar Problem FDT}
        \begin{tabular}{|c|c|c|c|c|c|c|}
            \hline
                   & $x_1$ & $y_1$ & $y_2$ & $z_1$ & $z_2$ \\ \hline
            $D_1$  & 1     &       & 1     & 1     & 1     \\ \hline
            $D_2$  &       & 1     &       & 1     & 1     \\ \hline
            $F$    & 1     & 1     & 1     &       & 1     \\ \hline
            $G_1$  &       & 1     &       &       &       \\ \hline
            $G_2$  &       &       & 1     &       &       \\
            \hline
        \end{tabular}
 \label{t:FDT_simple}
\end{table}%

    Alexandrov and Lewis introduced a graph based syntax called Reconfigurability in
    MDO Problem Synthesis (REMS) which incorporates objectives and constraints
    into a graph, effectively combining FDT and DSM \cite{alexandrov2004}. REMS retains the square adjacency
    matrix from DSM, but by adding the objectives and constraints, it partially
    combines a traditional DSM with an FDT. This formulation allows REMS to represent data
    dependency between multiple analysis tools as well as between analysis tools and
    objective/constraint functions. Additionally, REMS addresses the need to
    transition between multiple solution strategies while maintaining a single consistent
    graph representation of the fundamental problem formulation. However, REMS does not provide
    a mechanism for inclusion of solvers or optimizers in the graph. This fundamentally limits REMS
    from describing the specifics of different solution strategies as applied to a specific problem.

Tosserams et al. developed a technique for problem partitioning named $\Psi$ that, like REMS, is based on a formal language specification \cite{Tosserams2010}.  The $\Psi$ method allows the user to specify how an MDO problem is to be decomposed into multiple optimization subproblems.  These subproblems may involve one or several analyses and multiple design variables.  Unlike REMS which builds subproblems automatically, $\Psi$ is intended to enable a manual process of subproblem construction based on the designers’ goals.  After subproblems are built, the language includes constructs to link them together to form into a system-level optimization problem statement.  $\Psi$ does not predispose the coordination strategy in the MDO architecture; any applicable approach can be used to achieve the required consistency among subproblems during the system level optimization.   Compiler and generator functions have been developed for $\Psi$ to facilitate a straightforward specification of coordination approaches for MDO frameworks.

    Lamb and Martins included objectives and constraint functions as nodes
    in an Extended DSM (XDSM) in order to capture a more complete description
    of solution strategies for MDAO problems \cite{Lambe2012}. Unlike REMS,
    XDSM also includes nodes for solvers and optimizers to enable complete
    definition of MDAO architectures. Martins uses XDSM to describe thirteen different
    optimization architectures in a survey paper that provides a novel
    classification of the different techniques \cite{Lambe2011}. With the
    additional information included in an XDSM, Lu and Martins applied both
    ordering and partitioning algorithms to an MDAO test problem termed the
    Scalable Problem \cite{Lu2012}.

    \begin{figure}
        \begin{center}
        \includegraphics[height=.25\textheight]{XDSM/simple}
        \caption{XDSM for Eq. \ref{eqn:simple}, with a Gauss-Seidel iteration
          and MDF solution architecture. \label{fig:XDSM_simple}}
        \end{center}
    \end{figure}

    Although XDSM captures many of the functional aspects of FDT, it
    requires the use of solver and optimizer blocks to represent
    the relationship between design variables and objectives/constraints.
    By introducing solver or optimizer blocks, XDSM automatically implies a solution strategy.
    The XDSM for Eq. \ref{eqn:simple} is given in Figure \ref{fig:XDSM_simple}.
    This diagram is shown with an assumed Gauss-Seidel iteration scheme and an MDF solution architecture.

    In this paper, we propose a new graph syntax that combines certain features of REMS and XDSM.
    Despite sharing some features with both of these other graph based approaches, the syntax
    proposed here has a fundamentally different goal which is complementary to both of them.
    REMS and XDSM provide excellent human readable formats for an MDAO problem description, whereas the
    syntax described in this paper is designed to provide an effective machine readable format for MDAO problem definition.
    In the interest of simplifying algorithmic operations on the graphs, the specified
    format is both more verbose and less visually informative. The benefit is a consistent
    and easily utilized graph from a algorithmic perspective.

\subsection{Requirements for a New Graph Syntax}
  \label{s:requirements}
  The goal of the graph-based syntax presented here is to enable the general
  structure of an MDAO problem to be described independently of any solution information,
  while still being able to accommodate the more specific case when a solution
  strategy is applied. In order to achieve this goal,
  the graph syntax needs to accommodate a number of MDAO problem constructs:
  \begin{itemize}
    \item Analysis tools and their interconnections
    \item Design variables, objectives, and constraints
    \item Coupling between analyses
    \item Multi-fidelity analyses
  \end{itemize}

  The syntax is intended to represent three phases of the design problem
  formulation process. In the initial problem definition phase, the specific
  analysis tools and design goals are identified. Next, a single problem
  formulation is identified that specifies design variables, constraints,
  objectives, analysis tools, and all other elements required to represent
  the overall MDAO problem. Lastly, a specific procedure for solving the problem
  is selected, e.g. an MDAO optimization architecture. Using
  the proposed graph syntax, the outcome of these phases can be represented with the following graphs:
  \begin{itemize}
    \item Maximal Connectivity Graph (MCG)
    \item Fundamental Problem Graph (FPG)
    \item Problem Solution Graph (PSG)
  \end{itemize}

  The \emph{maximal connectivity graph} represents the first phase of the problem
  formulation with all analysis tools being considered and all possible connections
  between them also present. The second phase of problem formulation results in the \emph{fundamental problem graph},
  which comprises only the analysis tools, objectives, constraints, and variables to solve the problem.
  The final phase results in a \emph{problem solution graph} which includes additional
  edges and nodes to represent the solution strategy being employed to solve the
  problem.

  Comparisons of the number and size of each of these graphs are depicted in Figs.~\ref{f:tree} and \ref{f:hourglass}.
  The tree diagram illustrates that it is generally possible to obtain
  multiple FPGs from a single maximal connectivity graph. The multiple FPGs may correspond to
  different down-selections of analysis tools, different connections between the tools,
  or both. Each down-selection reduces the number of possible FPGs that could be reached
  until only one remains. For a given FPG, however, different PSGs may be obtained by implementing
  different solution strategies. Considering the size of a graph to be the sum of all of its
  edges and nodes, the hourglass shape in Fig. \ref{f:hourglass} qualitatively illustrates how
  the FPG is obtained from the MCG by \emph{removing} nodes and edges,
  and the PSG is obtained from the FPG by \emph{adding} nodes and edges.
These additions correspond to optimizers, solvers, and compatibility constraints required in particular MDAO solution architectures.
\begin{figure}[htb!]
    \centering
    \subfigure[Number of possible graphs]{
    \includegraphics[width=1.75in]{images/tree}
    \label{f:tree}
    }
    \subfigure[Graph size]{
    \includegraphics[width=1.75in]{images/hourglass}
    \label{f:hourglass}
    }
  \caption{The relationship between the MCG, FPG, and PSG.}
  \end{figure}

\subsection{Potential Applications}

The syntax proposed here for defining graphs is partially motivated by the unique needs
in formulating very large problems that have thousands of design variables. Such problems are
large enough that many traditional design methods stumble simply because one person cannot maintain a complete view of the problem. Hence, as problems
grow, software tools become needed to manage the complexity of the problem itself.

Even for small problems, algorithmic exploration of problem structure can yield deeper understanding
of problems and more effective solution methods, as demonstrated by the work by Michelena and Papalambros \cite{Michelena1997} and Allison \cite{Allison2008} in applying FDT to study problem partitioning. The syntax and rule set described
in this paper
provide a foundation for the development of new algorithms for analyzing and modifying
problem structure. This foundation is intended to facilitate a standard view of
problem definition and automatic methods for investigating different  problem formulations within MDAO frameworks.

